\chapter{Gerêcia de requisitos}
Neste capítulo serão apresentados os requisitos elicitados, desde os épicos, localizados no mais alto nível do SAFe até as histórias de usuário no nível \textit{Team}, o nível mais abaixo no SAFe.

Para a gerência de requisitos, foi utilizada a ferramenta \textit{TargetProcess}, como foi proposto no trabalho 1.
\section{Portfolio}
O \textit{Portfolio} é o mais alto nível nível no SAFe, nesta etapa do trabalho o nível de portfólio é evidenciado pelas seguintes atividades: Analisar o contexto, Definir o tema de investimento, Elicitar e validar os épicos do negócio e priorizar um épicos. Atividades essas que foram realizadas utilizando algumas técnicas de elicitação como: Entrevistas, Workshops e Brainstorm.
\subsection{Requisitos elicitados}
\textbf{Tema de investimento} - Gestão de clientes da EletronJun

Este tema foi extraído a partir da necessidade de organização e gestão dos clientes da empresa, visando facilitar a comunicação, além de aumentar a produtividade e a organização da empresa.

\textbf{Épico 01} - Gerenciamento de clientes

Compreente todas as ações do cliente dentro do sistema.

\textbf{Épico 02} - Gerenciamento de pedidos

Contém um registro de todos os pedidos realizados pelo cliente.

\textbf{Épico 03} - Gerenciamento de pagamentos

Contém um registro de todos os pagamentos do cliente.

\textbf{Épico 04} - Centralização da comunicação

Engloba toda a comunicação existente entre o cliente e a empresa.

A seguir os épicos estão sendo detalhados através do template \textit{lightwaight business case}, que é mencionado no \ref{safe}.

\begin{table}[]
\centering
\caption{Épico 01}
\label{label-epico01}
\begin{tabular}{
>{\columncolor[HTML]{96FFFB}}r l}
\multicolumn{2}{c}{\cellcolor[HTML]{34CDF9}Gerenciamento de clientes}                                                                                                        \\
Para                 & A empresa e seus clientes                                                                                                                   \\
Quem                 & Faz a gestão dos clientes da empresa                                                                                                        \\
A                    & Sistema de gerenciamento de clientes                                                                                                        \\
É uma                & Ferramenta que faz a gestão dos clientes da empresa dentro do sistema                                                                       \\
Que                  & Centraliza e facilita a gestão dos clientes em uma só ferramenta, aumentando a organização e a produtividade da empresa                     \\
Diferente            & Da forma como é feita atualmente, onde não há um gerenciamento eficiente e a rastreabilidade dos clientes é nula                            \\
Nossa solução        & Um sistema único e modularizado, onde existirá o módulo de gestão de clientes da EletronJun                                                 \\
\multicolumn{2}{c}{\cellcolor[HTML]{34CDF9}Escopo}                                                                                                                 \\
Critérios de sucesso & Implementação de um único sistema para a gestão de clientes de forma que a organização aumente e consequentemente a produtividade da empresa \\
No escopo            & Sistema para a gestão dos clientes                                                                                                          \\
Fora do escopo       & Aprovação do sistema pelos clientes                                                                                                        
\end{tabular}
\end{table}

\begin{table}[]
\centering
\caption{Épico 02}
\label{label-epico02}
\begin{tabular}{
>{\columncolor[HTML]{96FFFB}}r l}
\multicolumn{2}{c}{\cellcolor[HTML]{34CDF9}Gerenciamento de pedidos}                                                                                                                                 \\
Para                 & A empresa e seus clientes                                                                                                                                            \\
Quem                 & Faz a gestão dos pedidos realizados pelos clientes da empresa                                                                                                                                 \\
A                    & Sistema de gerenciamento de pedidos                                                                                                                                  \\
É uma                & Ferramenta que faz a gestão dos pedidos realizados pelos clientes                                                                                                    \\
Que                  & Melhora a forma como os pedidos são feitos, centralizando-os em uma única ferramenta                     \\
Diferente            & Da forma como é feita atualmente, onde não há um gerenciamento eficiente e não é feito nenhum registro dos pedidos para uma consulta posterior                       \\
Nossa solução        & Um sistema único e modularizado, onde existirá o módulo de gestão de pedidos realizados pelos clientes da EletronJun                                                 \\
\multicolumn{2}{c}{\cellcolor[HTML]{34CDF9}Escopo}                                                                                                                                          \\
Critérios de sucesso & Implementação de um único sistema para a gestão de pedidos realizados pelos clientes de forma que a organização aumente e consequentemente a produtividade da empresa \\
No escopo            & Sistema para a gestão de pedidos realizados pelos clientes                                                                                                           \\
Fora do escopo       & Aprovação do sistema pelos clientes                                                                                                                                 
\end{tabular}
\end{table}

\begin{table}[]
\centering
\caption{Épico 03}
\label{label-epico03}
\begin{tabular}{
>{\columncolor[HTML]{96FFFB}}r l}
\multicolumn{2}{c}{\cellcolor[HTML]{34CDF9}Gerenciamento de pagamentos}                                                                                                                                                                                                                                             \\
Para                 & A empresa e seus clientes                                                                                                                                                                                                                                                        \\
Quem                 & Faz a gestão dos pagamentos realizados pelos clientes da empresa                                                                                                                                                                                                                 \\
A                    & Sistema de pagamentos                                                                                                                                                                                                                                                            \\
É uma                & Ferramenta que faz a gestão dos pagamentos realizados pelos clientes                                                                                                                                                                                                             \\
Que                  & Centraliza e facilita os pagamentos realizados pelos clientes, aumentando a organização e os lucros da empresa                                                                                                                                                                   \\
Diferente            & Da forma como é feita atualmente, onde o cliente envia um email para a empresa e a empresa envia um número de conta bancária para que o cliente faça um depósito                                                                                                                 \\
Nossa solução        & Um sistema único e modularizado, onde existirá o módulo de pagamentos realizados pelos clientes da EletronJun, visando disponibilizar mais opções de pagamento para o cliente, como cartão de crédito, boleto e PagSeguro, além de permitir uma gestão dos pagamentos realizados \\
\multicolumn{2}{c}{\cellcolor[HTML]{34CDF9}Escopo}                                                                                                                                                                                                                                                      \\
Critérios de sucesso & Implementação de um único sistema para o pagamento dos pedidos realizados pelos clientes de forma que aumente a organização e consequentemente os lucros da empresa                                                                                                               \\
No escopo            & Sistema para o pagamento de pedidos realizados pelos clientes                                                                                                                                                                                                                    \\
Fora do escopo       & Aprovação do sistema pelos clientes                                                                                                                                                                                                                                             
\end{tabular}
\end{table}

\begin{table}[]
\centering
\caption{Épico 04}
\label{label-epico04}
\begin{tabular}{
>{\columncolor[HTML]{96FFFB}}r l}
\multicolumn{2}{c}{\cellcolor[HTML]{34CDF9}Centralização da comunicação}                                                                                                          \\
Para                 & A empresa e seus clientes                                                                                                                     \\
Quem                 & Faz a comunicação cliente/empresa                                                                                                             \\
A                    & Sistema para a comunicação                                                                                                                    \\
É uma                & Ferramenta que facilita a comunicação entre a empresa e o cliente                                                                             \\
Que                  & Centraliza a comunicação cliente/empresa em uma única ferramenta aumentando assim a rastreabilidade e a organização                           \\
Diferente            & Da forma como é feita atualmente, onde a comunicação é feita por diversos meios diferentes e não há uma rastreabilidade da comunicação feita  \\
Nossa solução        & Um sistema único e modularizado, onde existirá um módulo para a comunicação entre o cliente e a empresa                                       \\
\multicolumn{2}{c}{\cellcolor[HTML]{34CDF9}Escopo}                                                                                                                   \\
Critérios de sucesso & Implementação de um único sistema para a comunicação cliente/empresa de forma que aumente a organização e consequentemente a produtividade da empresa \\
No escopo            & Sistema para comunicação cliente/empresa                                                                                                      \\
Fora do escopo       & Aprovação do sistema pelos clientes                                                                                                          
\end{tabular}
\end{table}