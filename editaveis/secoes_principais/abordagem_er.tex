\chapter[Abordagem da Engenharia de Requisitos]{Abordagem da Engenharia de Requisitos}
"Um conjunto de atividades utilizadas para identificar e comunicar a finalidade de um sistema de software, e o contexto no qual será usado. Assim, a ER tua como a ponte entre as necessidades reais dos usuários, clientes e outros grupos afetados por um sistema de software, e as potencialidades e oportunidades oferecidas pela tecnologia".(EASTERBROOK, 2004)

"Nos dias atuais, a necessidade de automatização de processos se faz cada vez mais presente. Processos são essenciais dentro de qualquer organização em qualquer área de atuação, inclusive em uma das áreas mais recentes, que a producão de software. A definição de processos de desenvolvimento de software vem com o objetivo de aumentar a produtividade e diminuir os riscos de um projeto".(VIEIRA, 2003)

"Com o grande crescimento da demanda na produção de software na atualidade os prazos estão cada vez mais curtos e a cobrança relacionada a qualidade estão cada vez maiores. A qualidade de um produto de software está diretamente ligada a melhor forma de atender as necessidades do cliente, o que torna o produto totalmente dinâmico. Uma vez que a natureza do produto é dinâmica existem grandes dificuldades no gerenciamento do desenvolvimento, por exemplo: a essência volátil dos requisitos do cliente".(TAVARES, 2008)

Dados estes fatores, foram criados vários processos dentro da engenharia de software, a fim de amenizar esses problemas, diminuir o tempo gasto e os custos de desenvolvimento. Por exemplo, os processos: cascata, interativo, espiral e incremental.

Cada processo se encaixa melhor em um determinado contexto, geralmente o processo da ER é adaptado de acordo com a equipe, com o cliente e com o projeto. Dado o contexto do trabalho a ser confeccionado nessa disciplina foram propostas duas abordagens, sendo elas: Scaled Agile Framework(SAFe) e o Rational Unified Process(RUP). A seguir daremos uma breve explicação sobre como funciona cada abordagem e será exposta a justificativa da escolha da abordagem e suas atividades.
  \section{SAFe}
O Scaled Agile Framework (SAFe) é um processo que provê uma receita para a aplicação de práticas e princípios ágeis e enxutos em uma escala empresarial. Assim como o Scrum está para uma equipe ágil o SAFe está para um empreendimento ágil. Possui uma abordagem focada na gestão e captura dos requisitos, em vez de no design e implementação.

É composto essencialmente por três níveis: o nível de Time, o nível de Programa e o nível de Portfólio. Para cada nível, existe um nível diferente de narrativa de requisitos: épicos para o Portfólio, features para o Programa e histórias de usuários para o Time. A partir da versão 4.0 deste framework, foi adicionado um novo nível, facultativo, chamado Fluxo de Valor.

O SAFe possui nove princípios fundamentais, baseados nos princípios ágeis, que orientam os papéis e práticas que o tornam efetivo. Esses princípios são:
\begin{itemize}
\item Tenha uma visão econômica
\item Tenha um pensamento sistêmico
\item Assuma variabilidade e preserve opções
\item Construa incrementalmente com ciclos de aprendizagem integrados e rápidos
\item Baseie os marcos em avaliação objetiva dos sistemas 	
\item Descentralize a tomada de decisão
\item Desbloqueie a motivação intrínseca aos profissionais do conhecimento
\item Tenha cadência; sincronize com planejamento cross-domain
\item Visualize e limite o trabalho em progresso, reduza a carga e gerencie o comprimento da fila
\end{itemize}
\subsection{Big picture do SAFe}
  \begin{figure}[!htbp]
    \centering
    \includegraphics[scale=0.35]{editaveis/figuras/SAFe_big_picture}
    \caption[Big Picture do SAFe]{Big Picture do SAFe. \footnotemark}
    \label{big-picture-safe}
  \end{figure}
  \newpage

  \section{RUP}
  O Rational Unified Process (RUP) é um processo de software iterativo e incremental desenvolvido no final dos anos 90, para lidar com aplicações de larga escala robustas, escaláveis e extensíveis. Foi o primeiro processo largamente adotado que reconheceu a necessidade de estender a área de abrangência das atividades que ocorriam nas fases de iniciação, elaboração, construção e transição durante o ciclo de vida.[LEFFINGEL?] Possui uma abordagem baseada em disciplinas na atribuição de tarefas e responsabilidades em um empreendimento. [WTHREEX]
    \begin{figure}[!htbp]
    \centering
    \includegraphics[scale=3.0]{editaveis/figuras/Fases_do_RUP_-_portugues}
    \caption[Fases do RUP]{Fases do RUP. \footnotemark}
    \label{fases-rup}
  \end{figure}
Este processo é composto de duas dimensões:
\begin{itemize}
\item O eixo horizontal representa o tempo dividido nas fases do processo 	(Iniciação, Elaboração, Construção e Transição). Diz 	respeito ao aspecto dinâmico do processo quando ele é aprovado.
\item O eixo vertical representa os aspectos estáticos do processo, as 	disciplinas, fluxos de trabalho, papéis, atividades e artefatos. 	[WTHREEX]
\end{itemize}

O objetivo do RUP é produzir software de alta qualidade atendendo as necessidades do projeto em um cronograma e orçamento previsíveis.
  \section{Modelo de maturidade}
  Os modelos de maturidade de processos são um referencial usado para:
\begin{itemize}
\item Avaliar a capacidade de processos na realização de seus objetivos
\item Localizar oportunidades de melhoria de produtividade e qualidade e de reducão de custos
\item Planejar e monitorar as ações de melhoria contínua dos processos empresariais
\end{itemize}

  \subsection{Modelo CMMI(Capability Maturity Model Integration)}
  O CMMI é um modelo de referência que define práticas sejam elas genéricas ou específicas necessárias para o desenvolvimento e avaliação de maturidade no desenvolvimento de softwares em uma organização. As práticas que são abordadas neste modelo são: gerenciamento de requisitos, manipulação de riscos, medição de desempenho, planejamento de trabalho, tomada de decisão, entre outros. O modelo CMMI não pode ser considerado uma metodologia, pois não orienta como deve ser feito, e sim o que deve ser feito. Esse modelo foi desenvolvido pelo SEI(Software Engineering Institute) da Universidade Carnegie Mellon. É uma evolução do CMM, que foi baseado em algumas das ideias mais importantes dos movimentos de qualidade industrial das últimas décadas.
No CMMI uma organização opta por duas representações para a melhoria dos seus processos: Por estágios e contínua.
  \begin{figure}[!htbp]
    \centering
    \includegraphics[scale=0.5]{editaveis/figuras/cinco-niveis-maturidade-cmmi}
    \caption[Níveis de Maturidade]{Níveis de Maturidade. \footnotemark}
    \label{cinco-niveis-maturidade-cmmi}
  \end{figure}

  \subsection{Modelo MPS-BR (Melhoria de Processo do Software Brasileiro)} \label{mps-br}
MPS-BR significa Melhoria de Processo do Software Brasileiro, criado pelo Softex e patrocinado pelo MCT. O modelo de maturidade de processos e desenvolvimento de software conhecido como CMMI-DEV foi adaptado para empresas brasileiras, em especial para micro, pequenas e médias empresas, dando origem ao MPS-BR. Essa adaptação foi necessária por que o CMMI-DEV prevê o amadurecimento dos processos em apenas cinco níveis.

E com o passar do tempo percebeu-se a necessidade de uma funcionalidade mais gradual aqui no Brasil, por isso foi quebrado os cinco níveis do CMMI-DEV em sete, com vemos na figura a seguir:
  \begin{figure}[!htbp]
    \centering
    \includegraphics[scale=0.5]{editaveis/figuras/650x432xniveis_de_qualificacao}
    \caption[Níveis de Maturidade do MPS-Br]{Níveis de Maturidade do MPS-Br. \footnotemark}
    \label{niveis-maturidade-mps-br}
  \end{figure}
  
Como ilustrado na imagem, o MPS-Br possui possui uma divisão semelhante ao CMMI, entretando o mesmo encontra-se dividido nos níveis constituintes do A ao G, sendo o nível A o mais alto qualitativamente e o G o nível inicial(MPS de Software, São Paulo: SOFTEX, 2012).	
\begin{description}
\item[A.] Em otimização
\item[B.] Gerenciado quantitativamente
\item[C.] Definido
\item[D.] Largamente definido
\item[E.] Parcialmente definido
\item[F.] Gerenciado
\item[G.] Parcialmente gerenciado	
\end{description}			

Os processos referentes a engenharia de requisitos estão descritos no níveis, G e D, correspondentes aos processos de "Gerência de requisitos respectivamente" e "Desenvolvimento de requisitos".

\textbf{Gerência de requisitos}

Como o nome já diz, tem como objetivo gerenciar os requisitos do produto e dos componentes do produto/projeto, além de identificar inconsistências entre requisitos, planos e produtos de trabalho do projeto.

\textbf{Resultados esperados:}
\begin{itemize}
\item \textbf{GRE 1} - O entendimento dos requisitos é obtido junto aos fornecedores de requisitos.
\item \textbf{GRE 2} - Os requisitos são avaliados com base em critérios e objetivos e um comprometimento da equipe técnica com estes requisitos é obtido.
\item \textbf{GRE 3} - A rastreabilidade bidirecional entre os requisitos e os produtos de trabalho é estabelecida e mantida.
\item \textbf{GRE 4} - Revisões em planos e produtos de trabalho do projeto são realizadas visando identificar e corrigir inconsistências em relação aos requisitos.
\item \textbf{GRE 5} - Mudanças nos requisitos são gerenciadas ao longo do projeto.
\end{itemize}

\textbf{Desenvolvimento de requisitos:}
Visa definir os requisitos do cliente, do produto e dos componentes do produto.

\textbf{Resultados esperados:}
\begin{itemize}
\item \textbf{DRE 1} - As necessidades, expectativas e restrições do cliente, tanto do produto quanto de suas interfaces, são identificadas.
\item \textbf{DRE 2} - Um conjunto definido de requisitos do cliente é especificado e priorizado a partir das necessidades, expectativas e restrições identificadas.
\item \textbf{DRE 3} - Um conjunto de requisitos funcionais e não-funcionais, do produto que descrevem a solução do problema a ser resolvido, é definido e mantido a partir dos requisitos do cliente.
\item \textbf{DRE 4} - Os requisitos funcionais e não-funcionais de cada componente do produto são refinados, elaborados e alocados.
\item \textbf{DRE 5} - Interfaces internas e externas do produto e de cada componente do produto são definidas.
\item \textbf{DRE 6} - Conceitos operacionais e cenários são desenvolvidos.
\end{itemize}
  \section{Contexto dentro do projeto}
Neste projeto foi definida a utilização do modelo MPS-Br, levando em conta que ele é bem semelhante ao modelo CMMI e vai possibilitar a produção de resultados parecidos. Também foram observados outros critérios para a definição do modelo a ser utilizado.

Segundo a SOFTEX, mantenedora do MPS.BR, grande parte do modelo de maturidade é mantido em parte por instituições de ensino, isto garante uma integração permanente do modelo com o meio acadêmico, possibilitando a absorção dos avanços obtidos em pesquisas. É facilitada assim a aplicação do modelo em projetos desenvolvidos na academia, que, segundo vários especialistas, está alguns anos na frente do mercado. 

Assim, o MPS-Br foi a primeira opção destacada pelo grupo. Atentando-se também ao fato de ser um modelo que se adapta a realidade do desenvolvimento de software no Brasil. Como dito na seção \ref{mps-br}, o modelo foi pensado para ser utilizado no contexto de micro, pequenas e médias empresas, o que é virtualmente a característica da presente equipe, uma vez que o nosso cliente é uma empresa júnior de engenharia eletrônica. Outro fator que vale a pena destacar é o fato de que o MPS-Br possui uma maior divisão de níveis de maturidade, o que torna o processo suscetível a atingir níveis maiores de maturidade com maior simplicidade já que, essa característica de possuir mais níveis, confere um menor número de processos em cada nível, comparando-se com o CMMI.
\section{Mapeamento do contexto e abordagem}
Levando em consideração as diferentes abordagens que foram apresentadas, algumas características foram mapeadas de forma a elucidar a melhor escolha de processo a se fazer. Essas características estão relacionadas ao contexto da equipe, escopo do projeto, cliente e ambiente de trabalho.

\begin{itemize}
\item \textbf{A} - A pequena quantidade de pessoas na equipe, quatro membros, facilita seu auto gerenciamento.
\item \textbf{B} - O impacto resultante de uma atribuição funcional específica para um dos membros é extremamente relevante para uma equipe de tamanho reduzido e pode influenciar diretamente o sucesso do projeto principalmente em relação aos prazos a serem cumpridos.
\item \textbf{C} - Todos os membros da equipe já participaram de projetos os quais se utilizava uma abordagem ágil. Alguns nunca trabalharam de forma incisiva com metodologias tradicionais.
\item \textbf{D} - A equipe tem disponibilidade para se encontrar pelo menos 3 vezes na semana presencialmente e 7 vezes através de ferramentas com Google Hangouts.
\item \textbf{E} - O cliente possui grande disponibilidade para ser encontrado próximo ao nosso ambiente de trabalho. Seu representante, inclusive, participa de algumas disciplinas do curso de Engenharia de Software da Universidade de Brasília com os próprios membros da equipe.
\item \textbf{F} - A Empresa Júnior Eletronjun, apesar de ser a mais antiga das empresas juniores da Faculdade do Gama, ainda se encontra em processos de construção, estabelecimento no mercado, levantamento de cliente. Sendo assim, o escopo do projeto pode sofrer alterações substanciais no decorrer do desenvolvimento.
\item \textbf{G} - Um dos membros da presente equipe já trabalho na equipe de T.I e Marketing da Eletronjun nessa mesma gestão, o que favorece o diálogo e o reconhecimento do cliente por parte dos membros do grupo.
\item \textbf{H} - O projeto possui um tempo relativamente curto para ser implementado, com uma expectativa de 45 dias.
\item \textbf{I} - Os membros da equipe já se conheciam antes e trabalharam juntos, não todos, mas em parte, em outros projetos de disciplinas no curso de Engenharia de Software.
\end{itemize}			

  \section{Justificativa}
Conforme a seção anterior, a metodologia escolhida para a confecção deste trabalho foi uma abordagem ágil, fundamentada no SAFe.

Esta escolha foi baseada nas características da equipe, do projeto e do cliente, que automaticamente encontraram maior correspondência adaptativa em uma abordagem que segue os princípios do desenvolvimento ágil.

“A engenharia de software ágil combina filosofia com um conjunto de princípios de desenvolvimento. A filosofia defende a satisfação do cliente e a entrega de incremental prévio; equipes de projetos pequenas e altamente motivadas; métodos informais; artefatos de engenharia de software mínimos e, acima de tudo, simplicidade no desenvolvimento geral. Os princípios de desenvolvimento priorizam a entrega mais que a análise e projeto, embora essas atividades não sejam desencorajadas; também priorizam a comunicação ativa e contínua entre desenvolvedores e clientes”. (PRESSMAN, 2012)

Um outro fator muito importante para a escolha desta metodologia foi o prévio relacionamento que um dos membros do grupo tinha com o nosso cliente, uma vez que o mesmo já havia trabalhado na empresa, além de conhecer a sua estrutura, seus processos e parte da equipe. Isso tornou o processo mais fluido, uma vez que não tivemos a necessidade de fazer uma pesquisa minuciosa a fim de conhecer melhor a empresa.

Segundo Mike Cohn, autor do livro Desenvolvimento de Software com Scrum, uma equipe pequena consegue eficientemente uma ótima interação, facilitando o processo de desenvolvimento de um software, mas há um grande problema: como a equipe é pequena, qualquer perda ou comprometimento de um membro é potencialmente danoso ao projeto como um todo. Para que isso seja evitado, não deve haver grande especialização na equipe, ou seja, todos devem ser responsáveis pelo projeto, sem papéis fixos e inflexíveis, o que torna a equipe dinâmica e faz com que a metodologia se adapte ao escopo do problema.
