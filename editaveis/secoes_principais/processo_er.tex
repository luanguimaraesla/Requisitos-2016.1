\chapter[Processo da Engenharia de Requisitos]{Processo da Engenharia de Requisitos}
O processo descrito a seguir foi modelado na ferramenta Bizagi BPMN Modeler. Esta ferramenta foi escolhida após uma comparação com a ferramenta Bonita BPM, levando em conta critérios como popularidade, acessibilidade e funcionalidades disponíveis para o desenvolvimento do projeto.

Como dito anteriormente a abordagem a ser utilizada será uma abordagem ágil, fundamentada no SAFe (Scaled Agile Framework), utilizando os três níveis principais: Portfolio, Programa e Time. O nível de Fluxo de Valor não será utilizado por se entender que não há necessidade deste nível neste projeto.

A estrutura dos três níveis do SAFe foi mantida porque, segundo Leffingwell [\citeyear{leffingwell}], ao diminuir nível de abstração dos requisitos gradativamente, diminui também o nível de especificação precoce, reduzindo a sobrecarga ao gerenciar os requisitos. Isso também aumenta a agilidade do time por permitir a interpretação dos requisitos de maneira mais fácil para a implementação. O SAFe também prevê a gestão da rastreabilidade dos requisitos o que é de suma importância para a manutenção desse processo. Essa abordagem é explicada na Seção [SEÇÃO].

Um fator determinante para a aplicabilidade do SAFe a uma equipe mínima e um projeto de pequeno porte, é a consistência na seleção dos papeis que vão compor o processo. De acordo com a análise da viabilidade de tempo e recursos, do problema proposto e do cliente, foram estabelecidos os seguintes papéis:

\begin{description}
\item[Especialista do Negócio] É o \textit{stakeholder} que detém o conhecimento do negócio, do contexto organizacional e da visão do produto.    
\item[Product Owner (P.O.)] É o membro do time que fica responsável pela definição das histórias e pela priorização do \textit{Team Backlog}, além de participar do planejamento e validação da \textit{sprint} definindo os seus objetivos.
\item[Product Manager (P.M.)] De acordo com Leffingwell~[\citeyear{leffingwell}], cabe ao P.M.: manter a visão e o \textit{Program Backlog}, priorizar \textit{features}, manter o \textit{Roadmap}, gerenciar o conteúdo da \textit{Release} e manter e priorizar o \textit{Porfolio Backlog}. As atividades realizadas pelo P.M. acontecem nos níveis Portfolio e Programa.
\item[Scrum Master] Seu papel é dar assistência para o resto da equipe a fim de extrair a máxima perfomance, ele é de certa forma o líder do time~\cite{leffingwell}.
\item[Time] É composto por toda a equipe, desenvolvedores, \textit{designers} e etc.
\end{description}

\section{Big picture do processo}
  \begin{figure}[!htbp]
    \centering
    \includegraphics[scale=0.3]{figuras/Processo_v1-2}
    \caption[Big picture do processo.]{Big picture do processo. \footnotemark}
    \label{processo}
  \end{figure}

\section{Portfólio}
Esse nível tem como objetivo principal levantar e elaborar uma abstração de alto nível dos requisitos de negócio. Cada uma das atividades, artefatos gerados e papeis estabelecidos que serão descritos a seguir fazem parte da solução proposta para se resolver o problema existente. A execução de cada atividade será baseada na sistematização da análise documental e \textit{brainstormings}.

\subsection{Artefatos}
\begin{description}
\item[Contexto do cliente] É fornecido pelo cliente ou levantado a partir de outras fontes de informação que contenham dados sobre o próprio cliente e o problema a ser resolvido. Este artefato descreve resumidamente uma visão do contexto do problema e do cliente.
\item[Tema de investimento] Consiste na documentação de quais são os reais objetivos e valores da empresa e os benefícios que a solução em \textit{software} trará para a mesma. Por se tratar de um requisito de altíssimo nível é representado por uma breve descrição.
\item[Lista de Épicos] São iniciativas, ou frentes de negócio que podem gerar alto valor para o cliente, equipe ou o desenvolvimento de \textit{software} em si, como inovações arquiteturais ou implementação de tecnologias emergentes. Por se tratar de um requisito de alto nível, precisa ser trabalhado e refinado.
  \begin{enumerate}
    \item \textbf{Tempo:} um épico pode demandar tempo e vários ciclos de iteração para ser devidamente implementado.
    \item \textbf{Escopo:} por ser de alto nível afeta os subsequentes níveis do projeto, ocasionando um grande impacto e sendo uma área crítica pode custar muito para a equipe e cliente.
    \item \textbf{Negócio:} Representa a entrega de valor direta para o cliente, afetando diversos departamentos da área de negócios.
  \end{enumerate} 
\item[Backlog do Portfólio] Contém o detalhamento de cada um dos épicos e devidamente priorizados e organizados. Para isso, a cada épico é atribuida uma pontuação de acordo com sua complexidade seu valor gerado. Logicamente, para realizar uma avaliação precisa da complexidade da implementação do épico é necessário um detalhamento do mesmo.
\end{description}

\subsection{Descrição das atividades}

\newcolumntype{b}{>{\hsize=.7\hsize}X}
\newcolumntype{s}{>{\hsize=.3\hsize}X}

\begin{table}[!htbp]
\centering
\caption{Atividade: Analisar Contexto do Problema}
\label{atividade:1}
\begin{tabularx}{0.9\textwidth}{|>{\columncolor[HTML]{BBDAFF}}s |b|}
\hline
Identificador & 01                                                                  \\ \hline
Título        & Analisar contexto do problema                                       \\ \hline
Descrição     & Serão realizados encontros de forma que o cliente possa esclarecer o contexto no qual está inserido o problema, de forma que se busque um primeiro contato e o estabelecimento de um canal de diálogo da equipe com o cliente. \\ \hline
Entradas       & Contexto do Cliente                                                  \\ \hline
Saídas         & Temas de Investimentos                                               \\ \hline
Responsável   & Time                                                                  \\ \hline
\end{tabularx}
\end{table}

\begin{table}[!htbp]
\centering
\caption{Atividade: Definir tema de investimento}
\label{atividade:2}
\begin{tabularx}{0.9\textwidth}{|>{\columncolor[HTML]{BBDAFF}}s |b|}
\hline
Identificador & 02                                                                    \\ \hline
Título        & Definir tema de investimento                                          \\ \hline
Descrição     & Reunião entre o cliente e o gerente de portfolio a fim de definir o tema de investimento. \\ \hline
Entradas      &                                                                       \\ \hline
Saídas        & Tema(s) de investimento                                               \\ \hline
Responsável   & Gerente de portfolio                                                  \\ \hline
\end{tabularx}
\end{table}

\begin{table}[!htbp]
\centering
\caption{Atividade: Elicitar os épicos do negócio}
\label{atividade:3}
\begin{tabularx}{0.9\textwidth}{|>{\columncolor[HTML]{BBDAFF}}s |b|}
\hline
Identificador & 03                                                                    \\ \hline
Título          & Elicitar os épicos do negócio                                       \\ \hline
Descrição       & Criar uma lista de épicos pertinentes ao tema de investimento, isso poderá ser feito através de um brainstorm. \\ \hline
Entradas        & Tema de investimento                                                \\ \hline
Saídas        & Lista de épicos                                                       \\ \hline
Responsável   & Gerente de portfolio                                                  \\ \hline                                               
\end{tabularx}
\end{table}

\begin{table}[!htbp]
\centering
\caption{Atividade: Validar os épicos}
\label{atividade:4}
\begin{tabularx}{0.9\textwidth}{|>{\columncolor[HTML]{BBDAFF}}s |b|}
\hline
Identificador & 04                                                                    \\ \hline
Título          & Validar os épicos                                                   \\ \hline
Descrição       & Será realizada uma reunião entre o cliente e o gerente de portfolio com o objetivo de validar os épicos levantados. \\ \hline
Entradas        & Lista de épicos                                                     \\ \hline
Saídas          & Portfolio backlog                                                   \\ \hline
Responsável   & Gerente de portfolio                                                  \\ \hline
\end{tabularx}
\end{table}

\begin{table}[!htbp]
\centering
\caption{Atividade: Priorizar um épico}
\label{atividade:5}
\begin{tabularx}{0.9\textwidth}{|>{\columncolor[HTML]{BBDAFF}}s |b|}
\hline
Identificador & 05                                                                     \\ \hline
Título          & Priorizar um épico                                                   \\ \hline
Descrição      & O gerente de portfolio irá decidir em uma conversa com o cliente qual é o épico mais importante a ser realizado naquele momento do projeto.                                                                                    \\ \hline
Entradas      & Portfolio backlog                                                      \\ \hline
Saídas        & Épico priorizado                                                       \\ \hline
Responsável   & Gerente de portfolio                                                   \\ \hline
\end{tabularx}
\end{table}

\begin{table}[!htbp]
\centering
\caption{Atividade: Gerenciar os épicos}
\label{atividade:6}
\begin{tabularx}{0.9\textwidth}{|>{\columncolor[HTML]{BBDAFF}}s |b|}
\hline
Identificador & 06                                                                      \\ \hline
Título          & Gerenciar os épicos                                                   \\ \hline
Descrição      & O gerente de portfolio junto ao resto da equipe irão verificar se existem mais épicos no portfolio backlog ou se é necessária a criação de novos épicos.                                                                        \\ \hline
Entradas      & Portfolio backlog                                                       \\ \hline
Saídas        &                                                                         \\ \hline
Responsável   & Gerente de portfolio                                                    \\ \hline                                  
\end{tabularx}
\end{table}

\section{Programa}
Nessa etapa do processo se baseia na identificação de features e requisitos não-funcionais, na elaboração das correspondentes histórias de usuário e no planejamento das entregas do \textit{software}. Há também uma preocupação em se estabelecer estratégias para o desenvolvimento eficaz do \textit{software}.

\subsection{Artefatos}
\begin{description}
\item[Program Backlog] É o documento onde as \textit{features} são registradas, como uma pilha de \textit{features}.
\item[Roadmap] Consiste em uma série de \textit{releases} com datas planejadas, cada uma pertinente a um tema, um conjunto de objetivos e a priorização das \textit{features}. O \textit{Roadmap} nos dá uma ideia de como a equipe planeja mostrar valor no decorrer do tempo~\cite{leffingwell}.
\item[Visão] É um mecanismo utilizado para definir e comunicar diversos aspectos do sistema. Segundo Leffingwell~[\citeyear{leffingwell}], a visão de um sistema é composta por um conjunto de características que irão descrever as possibilidades do sistema, isto é, tudo aquilo que ele poderá oferecer ao usuário a fim de atender as necessidades dos envolvidos.
\end{description}

\clearpage

\subsection{Descrição das atividades}

\begin{table}[!htbp]
\centering
\caption{Atividade: Levantar \textit{features}}
\label{atividade:7}
\begin{tabularx}{0.9\textwidth}{|>{\columncolor[HTML]{BBDAFF}}s |b|}
\hline
Identificador & 07                                                                  \\ \hline
Título        & Levantar \textit{features}                                          \\ \hline
Descrição     & Consiste em levantar e detalhar a partir de técnicas conhecidas como reuniões e \textit{brainstormings} diversas funcionalidades que fazem parte do escopo do épico previamente selecionado.                                 \\ \hline
Entrada       & Épico da Iteração                                                   \\ \hline
Saída         & \textit{Program Backlog} e insumos para o Documento de Visão        \\ \hline
Responsável   & Time                                                                \\ \hline
\end{tabularx}
\end{table}

\begin{table}[!htbp]
\centering
\caption{Atividade: Levantar requisitos não-funcionais}
\label{atividade:8}
\begin{tabularx}{0.9\textwidth}{|>{\columncolor[HTML]{BBDAFF}}s |b|}
\hline
Identificador & 08                                                                   \\ \hline
Título        & Levantar requisitos não-funcionais                                  \\ \hline
Descrição     & A partir da realização de reuniões com os clientes e o time serão levantadas algumas condições para a construção da solução. Padrões de projeto, arquitetura, servidores, acessibilidade, etc.                               \\ \hline
Entrada       & \textit{Portfolio Backlog}                                          \\ \hline
Saída         & Insumos para o Documento de Visão                                   \\ \hline
Responsável   & Time                                                                \\ \hline
\end{tabularx}
\end{table}

\begin{table}[!htbp]
\centering
\caption{Atividade: Definir visão}
\label{atividade:9}
\begin{tabularx}{0.9\textwidth}{|>{\columncolor[HTML]{BBDAFF}}s |b|}
\hline
Identificador & 09                                                                   \\ \hline
Título        & Definir visão                                                       \\ \hline
Descrição     & A partir dos insumos fornecidos pelas diversas reuniões e de um pensamento alinhado da equipe e cliente sobre o problema a ser resolvido,será elaborado um documento que contenha os fatores e os pontos de discussão mais relevantes para a concepção e construção do \textit{software}.                                                                                              \\ \hline
Entrada       & \textit{Backlog do Programa}                                        \\ \hline
Saída         & Documento de Visão                                                  \\ \hline
Responsável   & Time                                                                \\ \hline
\end{tabularx}
\end{table}

\clearpage

\begin{table}[!htbp]
\centering
\caption{Atividade: Priorizar \textit{features}}
\label{atividade:1}
\begin{tabularx}{0.9\textwidth}{|>{\columncolor[HTML]{BBDAFF}}s |b|}
\hline
Identificador & 10                                                                  \\ \hline
Título        & Priorizar \textit{features}                                         \\ \hline
Descrição     & Com as \textit{features} definidas, cada uma deve receber uma pontuação de acordo com sua relevância, ou seja, valor que agrega para o cliente e dificuldade de implementação.                                                \\ \hline
Entrada       & \textit{Backlog do Programa}                                        \\ \hline
Saída         & \textit{Roadmap}                                                    \\ \hline
Responsável   & Time                                                                \\ \hline
\end{tabularx}
\end{table}


\section{Time}
O nível da equipe descreve como as equipes ágeis aplicam o SAFe integrado com as práticas Scrum/XP e análise de qualidade para garantir a entrega da solução descrita nos documentos de requisitos.

\subsection{Artefatos}
\begin{description}
\item[Team Backlog] Representa uma coleção de tudo que o time necessita para implementar aquela parte do sistema. Pode conter histórias de usuário ou \textit{enabler histories}. Sendo que a maioria tem origem no \textit{Program Backlog}, enquanto algumas são pertinentes a um contexto específico do time. O \textit{Team Backlog} pertence ao P.O., vale salientar que o fato de "pertencer" ao P.O. não significa que ele é o único que pode alterá-lo, mas é preferível que seja ele a fazer isso. (REFERÊNCIA SAFe, 2015)
\item[Sprint Backlog] É uma lista de tarefas que o time se compromete a fazer em uma \textit{sprint}. Os itens do \textit{Sprint Backlog} são extraídos do \textit{Team Backlog}, com base nas prioridades previamente definidas. É a percepção da equipe sobre o tempo que será necessário para completar um conjunto de funcionalidades.
\item[Build] É o incremento gerado a cada \textit{sprint}.
\end{description}

\clearpage

\subsection{Descrição das atividades}

\begin{table}[!htbp]
\centering
\caption{Atividade: Planejar \textit{sprint}}
\label{atividade:1}
\begin{tabularx}{0.9\textwidth}{|>{\columncolor[HTML]{BBDAFF}}s |b|}
\hline
Identificador & 15                                                                  \\ \hline
Título        & Planejar \textit{sprint}                                            \\ \hline
Descrição     & Após análise de cada item do \textit{Team Backlog}, o time estima o esforço necessário para a realização de cada item. Então sabendo o que é prioritário para o \textit{Product Owner}, define-se o que será realizado na \textit{sprint} em questão, formando o \textit{Sprint Backlog}.\\ \hline
Entrada       & \textit{Team Backlog}                                               \\ \hline
Saída         & \textit{Sprint Backlog}                                             \\ \hline
Responsável   & \textit{Scrum Master}, Time                                         \\ \hline
\end{tabularx}
\end{table}

\begin{table}[!htbp]
\centering
\caption{Atividade: Executar \textit{sprint}}
\label{atividade:1}
\begin{tabularx}{0.9\textwidth}{|>{\columncolor[HTML]{BBDAFF}}s |b|}
\hline
Identificador & 16                                                                  \\ \hline
Título        & Executar \textit{sprint}                                            \\ \hline
Descrição     & Representa um ciclo de trabalho. As \textit{sprints} devem ter sempre a mesma duração, uma semana.” A cada \textit{sprint} um conjunto de requisitos é implementado, tendo como resultado um incremento do produto que está sendo desenvolvido.\\ \hline
Entrada       & \textit{Sprint Backlog}                                             \\ \hline
Saída         & \textit{Build}                                                      \\ \hline
Responsável   & Time                                                                \\ \hline
\end{tabularx}
\end{table}

\begin{table}[!htbp]
\centering
\caption{Atividade: Fazer revisão da \textit{sprint}}
\label{atividade:1}
\begin{tabularx}{0.9\textwidth}{|>{\columncolor[HTML]{BBDAFF}}s |b|}
\hline
Identificador & 17                                                                  \\ \hline
Título        & Fazer revisão da \textit{sprint}                                    \\ \hline
Descrição     & Aqui é revisado com o Product owner o que foi gerado ao produto após a execução da sprint. Os itens do Sprint Backlog que não foram gerados são adiados para a próxima sprint.                                               \\ \hline
Entrada       & \textit{Team Backlog}                                               \\ \hline
Saída         & Resumo da revisão da \textit{sprint}                                \\ \hline
Responsável   & Time, \textit{Scrum Master}, \textit{Product Owner}                 \\ \hline
\end{tabularx}
\end{table}

\begin{table}[!htbp]
\centering
\caption{Atividade: Fazer retrospectiva da \textit{sprint}}
\label{atividade:1}
\begin{tabularx}{0.9\textwidth}{|>{\columncolor[HTML]{BBDAFF}}s |b|}
\hline
Identificador & 18                                                                  \\ \hline
Título        & Fazer revisão da \textit{sprint}                                    \\ \hline
Descrição     & Nesta atividade, a equipe se reúne para levantar as práticas que deram certo na \textit{sprint}, o que deu errado, e o que pode ser feito para melhorar.                                                                     \\ \hline
Entrada       & Resumo da Revisão da \textit{sprint}                                \\ \hline
Saída         & Resumo da Retrospectiva da \textit{sprint}                          \\ \hline
Responsável   & Time, \textit{Scrum Master}                                         \\ \hline
\end{tabularx}
\end{table}
