\chapter[Processo da Engenharia de Requisitos]{Processo da Engenharia de Requisitos}
O processo descrito a seguir foi modelado na ferramenta Bizagi BPMN Modeler. Esta ferramenta foi escolhida após uma comparação com a ferramenta Bonita BPM, levando em conta critérios como popularidade, acessibilidade e funcionalidades disponíveis para o desenvolvimento do projeto.

Como dito anteriormente a abordagem a ser utilizada será uma abordagem ágil, fundamentada no SAFe (Scaled Agile Framework), utilizando os três níveis principais: Portfolio, Programa e Time. O nível de Fluxo de Valor não será utilizado por se entender que não há necessidade deste nível neste projeto.

A estrutura dos três níveis do SAFe foi mantida porque, segundo Leffingwell (2011), ao diminuir nível de abstração dos requisitos gradativamente, diminui também o nível de especificação precoce, reduzindo a sobrecarga ao gerenciar os requisitos. Isso também aumenta a agilidade do time por permitir a interpretação dos requisitos de maneira mais fácil para a implementação.

A gestão da rastreabilidade dos requisitos é de suma importância para a manutenção desse processo. Todavia, o SAFe não define consistentemente a manutenibilidade entre seus três níveis, refletindo em uma escolha da equipe por adotar atividades de gerência de mudanças do RUP. Essa abordagem é explicada na Seção [SEÇÃO]. É importante dizer que o SAFe considera sim a mutabilidade dos requisitos e que essa escolha é uma maneira encontrada para lidar com essas mudanças e seus respectivos riscos da melhor forma possível.

Um fator determinante para a aplicabilidade do SAFe a uma equipe mínima e um projeto de pequeno porte, é a consistência na seleção dos papeis que vão compor o processo. De acordo com a análise da viabilidade de tempo e recursos, do problema proposto e do cliente, foram estabelecidos os seguintes papéis:

\begin{description}
\item[Especialista do Negócio] É o stakeholder que detém o conhecimento do negócio, do contexto organizacional e da visão do produto.    
\item[Product Owner (P.O.)] É o membro do time que fica responsável pela definição das histórias e pela priorização do Team Backlog, além de participar do planejamento e validação da sprint definindo os seus objetivos.
\item[Product Manager (P.M.)] De acordo com Leffingwell(2011), cabe ao P.M.: manter a visão e o program backlog, priorizar features, manter o roadmap, gerenciar o conteúdo da release e manter e priorizar o porfolio backlog. As atividades realizadas pelo P.M. acontecem nos níveis Portfolio e Program.
\item[Scrum Master] Seu papel é dar assistência para o resto da equipe a fim de extrair a máxima perfomance, ele é de certa forma o líder do time(Leffingwell, 2011).
\item[Time] É composto por toda a equipe, desenvolvedores, designers e etc.
\end{description}

\section{Big picture do processo}
  \begin{figure}[!htbp]
    \centering
    \includegraphics[scale=0.3]{figuras/Processo_v1-2}
    \caption[Big picture do processo.]{Big picture do processo. \footnotemark}
    \label{processo}
  \end{figure}
\section{Papéis}

\section{Artefatos}
\textbf{Temas de investimento} - Representam um conjunto de iniciativas guias para o investimento da instituição, seja em sistemas, produtos, aplicações ou serviços(Leffingwell, 2011).

\textbf{Épicos} - São iniciativas de desenvolvimento em larga escala que agregam valor a um tema de investimento(Leffingwell, 2011). São os de requisitos que possuem o mais alto nível no processo(Leffingwell, 2011).

\textbf{Portfolio backlog} - É o local onde os épicos são registrados, como um repositório de épicos.

\textbf{Features} - Atuam como pontes entre as necessidades dos stakeholders e os requisitos específicos no domínio da solução(Leffingwell, 2011).

\textbf{Program backlog} - É o local onde as features são registradas, como um repositório de features.

\textbf{Roadmap} - Consiste em uma série de releases com datas planejadas, cada uma pertinente a um tema, um conjunto de objetivos e uma feature priorizada. O roadmap nos dá uma ideia de como a instituição planeja mostrar valor com o decorrer do tempo(Leffingwell, 2011).

\textbf{Visão} - É um mecanismo utilizado para definir e comunicar a visão do sistema(Leffingwell, 2011). A visão de um sistema é composta por um conjunto de features que irão descrever as possibilidades do sistema, isto é, tudo aquilo que ele poderá oferecer ao usuário a fim de atender as necessidades dos envolvidos(Leffingwell, 2011).

\textbf{Team backlog} - Representa uma coleção de tudo que o time necessita para progredir aquela porção do sistema. Pode conter histórias de usuário ou enabler histories. Sendo que a maioria tem origem no program backlog, enquanto algumas são pertinentes a um contexto específico do time. O team backlog pertence ao PO, vale salientar que o fato de "pertencer" ao PO não significa que ele é o único que pode alterá-lo, mas é preferível que seja ele a fazer isso. (SAFe, 2015)

\textbf{Sprint backlog} - É o local onde serão armazenadas as histórias de usuário a serem realizadas naquela sprint.

\textbf{Build} - É o incremento de software gerado em cada sprint.

\section{Atividades}
