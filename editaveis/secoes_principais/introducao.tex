\chapter[Introdução]{Introdução}
  \section{Contexto do cliente}
A empresa júnior de engenharia eletrônica da Universidade de Brasília - Faculdade do Gama, a Eletrojun, foi a primeira empresa júnior da FGA. Por ser a empresa mais antiga ela é a que tem maior demanda de serviços e o maior número de clientes. Isso gerou uma carência no que diz respeito ao gerenciamento de clientes e seus respectivos projetos. Desde o primeiro contato com o cliente até a entrega do produto final, o projeto passa por várias etapas em diferentes setores da empresa. Isso requer um processo minucioso, que se atente com cada detalhe, cada iteração.

A Eletrojun possui uma organização estrutural linear e vertical. No topo da empresa está o presidente, logo abaixo vem os diretores, sendo eles: administrativos, financeiros, de gestão, marketing e projetos. Abaixo dos diretores estão os gerentes, dispostos em 3 áreas principais, que são: administrativa, financeira e de auditoria. Abaixo da gerência estão todos os outros membros da instituição. Este tipo de estrutura torna bem claro para cada setor quais atividades devem ser desenvolvidas.