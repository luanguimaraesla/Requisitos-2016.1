\chapter{Planejamento da Primeira Iteração}
Este capítulo trata das histórias de usuário que foram implementadas e sua devida hierarquia(de história até os épicos). Cada história de usuário contém:
\begin{itemize}
\item \textbf{Título}
\item \textbf{Valor de negócio, sendo classificado em 5 níveis}: Deve ter, ótimo, bom, médio e bom ter. O seu nível de importãncia é a respectiva ordem de apresentação, ou seja, deve ter(mais importante) e bom ter(menos importante)
\item \textbf{Quem}: Indicando o agente
\item \textbf{O que}: Indicando o que faz
\item \textbf{Porque}: Indicando o objetivo
\item \textbf{Critérios de aceitação}: Indicando formas de usar a funcionalidade implementada em uma história
\item \textbf{Pontos}: Indicando o grau de dificuldade de implementação da história
\end{itemize}

\begin{enumerate}
	\item \textbf{Épico}: Gerenciamento de clientes
	\begin{enumerate}
		\item \textbf{Feature}: Manutenção de clientes
		\begin{enumerate}
		\item \textbf{História de usuário}: 01
		\end{enumerate}
	\end{enumerate}
\end{enumerate}

\begin{table}[!hb]
\centering
\label{h01}
\begin{tabular}{|l|r|}
\hline
Título: Cadastro de cliente                                                                          & Valor de negócio: Deve ter                                                                         \\ \hline
\multicolumn{2}{|l|}{Quem: Eu como cliente}                                                                                                                                                               \\ \hline
\multicolumn{2}{|l|}{O que: Gostaria de criar uma conta pessoal}                                                                                                            \\ \hline
\multicolumn{2}{|l|}{Porque: Para gerenciar os meus acessos e atividades.}                                                                                                                                \\ \hline
\multicolumn{2}{|l|}{\begin{tabular}[c]{@{}l@{}}Critérios de aceitação:\\ 1. O cadastro só poderá ser efetuado caso não exista nenhuma \\outra conta atrelada ao mesmo CPF/CNPJ\end{tabular}} \\ \hline
\multicolumn{2}{|r|}{Pontos:}                                                                                                                                                                             \\ \hline
\end{tabular}
\caption{História de usuário 01}
\end{table}

\begin{enumerate}
	\item \textbf{Épico}: Gerenciamento de clientes
	\begin{enumerate}
		\item \textbf{Feature}: Manutenção de clientes
		\begin{enumerate}
		\item \textbf{História de usuário}: 02
		\end{enumerate}
	\end{enumerate}
\end{enumerate}

\begin{table}[!hbt]
\centering
\label{h02}
\begin{tabular}{|l|r|}
\hline
Título: Manter cliente                                                         & Valor de negócio: Deve ter                                                        \\ \hline
\multicolumn{2}{|l|}{Quem: Eu como cliente}                                                                                                                        \\ \hline
\multicolumn{2}{|l|}{O que: Gostaria de gerenciar a minha conta}                                                                                                   \\ \hline
\multicolumn{2}{|l|}{Porque: Para manter a minha conta e meus dados atualizados}                                                                                   \\ \hline
\multicolumn{2}{|l|}{\begin{tabular}[c]{@{}l@{}}Critérios de aceitação:\\ 1. O cliente deverá ter uma conta ativa\\ 2. O cliente deverá estar logado\end{tabular}} \\ \hline
\multicolumn{2}{|r|}{Pontos:}                                                                                                                                      \\ \hline
\end{tabular}
\caption{História de usuário 02}
\end{table}

\begin{enumerate}
	\item \textbf{Épico}: Gerenciamento de pedidos
	\begin{enumerate}
		\item \textbf{Feature}: Manutenção de pedidos
		\begin{enumerate}
		\item \textbf{História de usuário}: 03
		\end{enumerate}
	\end{enumerate}
\end{enumerate}

\begin{table}[!hb]
\centering
\label{h03}
\begin{tabular}{|l|r|}
\hline
Título: Fazer orçamento                                                           & Valor de negócio: Bom                                                          \\ \hline
\multicolumn{2}{|l|}{Quem: Eu como cliente}                                                                                                                        \\ \hline
\multicolumn{2}{|l|}{O que: Gostaria de um enviar um projeto para fazer um orçamento}                                                                              \\ \hline
\multicolumn{2}{|l|}{Porque: Para ter uma noção do custo do projeto}                                                                                               \\ \hline
\multicolumn{2}{|l|}{\begin{tabular}[c]{@{}l@{}}Critérios de aceitação:\\ 1. O cliente deverá ter uma conta ativa\\ 2. O cliente deverá estar logado\end{tabular}} \\ \hline
\multicolumn{2}{|r|}{Pontos:}                                                                                                                                      \\ \hline
\end{tabular}
\caption{História de usuário 03}
\end{table}

\begin{enumerate}
	\item \textbf{Épico}: Gerenciamento de pedidos
	\begin{enumerate}
		\item \textbf{Feature}: Manutenção de pedidos
		\begin{enumerate}
		\item \textbf{História de usuário}: 04
		\end{enumerate}
	\end{enumerate}
\end{enumerate}

\begin{table}[!hb]
\centering
\label{h04}
\begin{tabular}{|l|r|}
\hline
Título: Fazer pedido                                                          & Valor de negócio: Deve ter                                                         \\ \hline
\multicolumn{2}{|l|}{Quem: Eu como cliente}                                                                                                                        \\ \hline
\multicolumn{2}{|l|}{O que: Gostaria de fazer um pedido}                                                                                                           \\ \hline
\multicolumn{2}{|l|}{Porque: Para que sejam realizados os meus projetos}                                                                                           \\ \hline
\multicolumn{2}{|l|}{\begin{tabular}[c]{@{}l@{}}Critérios de aceitação:\\ 1. O cliente deverá ter uma conta ativa\\ 2. O cliente deverá estar logado\end{tabular}} \\ \hline
\multicolumn{2}{|r|}{Pontos:}                                                                                                                                      \\ \hline
\end{tabular}
\caption{História de usuário 04}
\end{table}

\begin{enumerate}
	\item \textbf{Épico}: Gerenciamento de pedidos
	\begin{enumerate}
		\item \textbf{Feature}: Manutenção de pedidos
		\begin{enumerate}
		\item \textbf{História de usuário}: 05
		\end{enumerate}
	\end{enumerate}
\end{enumerate}

\begin{table}[!hb]
\centering
\label{h05}
\begin{tabular}{|l|r|}
\hline
Título: Gerenciar pedidos                                                                                 & Valor de negócio: Bom                                                                               \\ \hline
\multicolumn{2}{|l|}{Quem: Eu como cliente}                                                                                                                                                                     \\ \hline
\multicolumn{2}{|l|}{O que: Gostaria de gerenciar os pedidos realizados}                                                                                                                                        \\ \hline
\multicolumn{2}{|l|}{Porque: Para ter um histórico e melhor controle dos pedidos}                                                                                                                               \\ \hline
\multicolumn{2}{|l|}{\begin{tabular}[c]{@{}l@{}}Critérios de aceitação:\\ 1. O cliente deverá ter uma conta ativa\\ 2. O cliente deverá estar logado\\ 3. O cliente deverá ter feito algum pedido\end{tabular}} \\ \hline
\multicolumn{2}{|r|}{Pontos:}                                                                                                                                                                                   \\ \hline
\end{tabular}
\caption{História de usuário 05}
\end{table}

\begin{enumerate}
	\item \textbf{Épico}: Gerenciamento de pedidos
	\begin{enumerate}
		\item \textbf{Feature}: Manutenção de pedidos
		\begin{enumerate}
		\item \textbf{História de usuário}: 06
		\end{enumerate}
	\end{enumerate}
\end{enumerate}

\begin{table}[!hb]
\centering
\label{h06}
\begin{tabular}{|l|r|}
\hline
Título: Status do pedido                                                                                  & Valor de negócio: Bom                                                                                \\ \hline
\multicolumn{2}{|l|}{Quem: Eu como cliente}                                                                                                                                                                      \\ \hline
\multicolumn{2}{|l|}{O que: Gostaria de acompanhar o status do pedido}                                                                                                                                           \\ \hline
\multicolumn{2}{|l|}{Porque: Para saber o estado de desenvolvimento do projeto}                                                                                                                                  \\ \hline
\multicolumn{2}{|l|}{\begin{tabular}[c]{@{}l@{}}Critérios de aceitação:\\ 1. O cliente deverá ter uma conta ativa\\ 2. O cliente deverá estar logado\\ 3. O cliente deverá ter um pedido em aberto\end{tabular}} \\ \hline
\multicolumn{2}{|r|}{Pontos:}                                                                                                                                                                                    \\ \hline
\end{tabular}
\caption{História de usuário 06}
\end{table}