\chapter[Considerações finais]{Considerações finais}

A partir dos estudos realizados para a composição consistente desse documento, torna-se clara e indiscutível a necessidade da aplicação da ciência proporcionada pela Engenharia de Requisitos. Este é o primeiro passo para o desenvolvimento de um projeto e, por isso, qualquer escolha pode ser considerada de impacto relevante sobre o sucesso do mesmo.

Delimitar com consciência e apoiado por ferramentas e técnicas o escopo de um projeto é, em sua essência, um dos fatores primordiais para o êxito de projetos de \textit{software}. Esse processo está totalmente embasado nas considerações dos diversos fatores como a equipe de desenvolvimento, o tempo disponível, os recursos econômicos, o perfil do cliente, etc.

Não foi possível se verificar empiricamente o impacto das discussões apresentadas nesse documento, entretanto, através das bibliografias neste contidas, toma-se o conhecimento de uam série de benefícios para os envolvidos ao se aplicar técnicas adequadas na seleção e aplicação de um processo.

A seleção de um processo e sua adequação ao escopo do projeto são etapas que se tratadas com a devida importância e precaução, facilitarão a desenvoltura dos trabalhos e atividades levantadas e, consequentemente, sua conclusão com primazia. É importante para um Engenheiro de Software evoluir sua percepção do problema dentro de seu contexto, para propor as melhores abordagens e almejar as prováveis solucoes.

Não obstante, a Engenharia de Requisitos não se baseia apenas em delimitar o escopo e o processo. Como foi mostrado nesse documento, há ainda a necessidade de se estabelecer um processo de obtenção desses requisitos, através do uso de técnicas de elicitação, estratégias de rastreabilidade, etc. Dessa forma, elimina-se uma série de problemas relacionados a obtenção de requisitos, que como já foi dito, é uma atividade de alto risco para o sucesso do projeto. Assim, o objetivo torna-se garantir que o produto de software realmente atenda as necessidades e objetivos do cliente.  
